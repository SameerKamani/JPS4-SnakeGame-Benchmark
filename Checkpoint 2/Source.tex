\documentclass[11pt]{article}
\usepackage[utf8]{inputenc}
\usepackage[T1]{fontenc}
\usepackage{lmodern}
\usepackage{amsmath, amssymb}
\usepackage{geometry}
\usepackage{enumitem}
\usepackage{hyperref}
\usepackage{parskip}
\usepackage{titlesec}
\usepackage{graphicx}
\usepackage{float}

\geometry{letterpaper, margin=1in}

\titleformat{\section}
  {\normalfont\Large\bfseries}{\thesection}{1em}{}
\titleformat{\subsection}
  {\normalfont\large\bfseries}{\thesubsection}{1em}{}

\title{Technical Summary:\\
\textbf{Jump Point Search Pathfinding in 4-connected Grids} \\
(Summary of Johannes Baum's Paper)}
\author{Team 66 - L1}
\date

\begin{document}

\maketitle
\thispagestyle{empty}

\section*{Overview}
This document summarizes the key ideas and challenges in implementing Johannes Baum’s paper, \textit{Jump Point Search Pathfinding in 4-connected Grids}. Our goal is to understand how JPS4 works, why it is needed, and what challenges we might face during its implementation.

\section{Problem and Contribution}
Traditional algorithms like A* often waste time by expanding many unnecessary nodes in dense grid maps. Although JPS8 improves efficiency in 8-connected grids, its techniques do not directly translate to 4-connected grids, where movement is limited to four cardinal directions. This is where \textbf{JPS4} comes in.

JPS4 adapts the jump point search to 4-connected grids by:
\begin{itemize}
    \item Using a horizontal-first \textbf{canonical ordering} to eliminate redundant paths.
    \item \textbf{Pruning non-essential neighbors} and only considering \emph{forced neighbors} when obstacles force a deviation.
    \item Introducing \textbf{jump points} at key obstacle corners, allowing the search to bypass large sections of nodes.
\end{itemize}
These improvements result in significantly faster pathfinding in cluttered environments such as video game maps and robotics navigation.

Before we dive deeper, Figure~\ref{fig:canonical} from the paper illustrates how multiple optimal paths in an obstacle-free grid are pruned into one unique, canonical path.

\begin{figure}[H]
    \centering
    \includegraphics[width=0.8\textwidth]{figure1.png}
    \caption{Canonical ordering in an obstacle-free grid. Dashed lines indicate multiple optimal paths, which are pruned to a single canonical path using a horizontal-first strategy.}
    \label{fig:canonical}
\end{figure}

\section{How the Algorithm Works}
JPS4 builds on a simple but powerful idea: only explore what’s necessary.

\subsection*{Canonical Ordering and Neighbor Pruning}
By favoring horizontal moves, the algorithm defines a unique, canonical path. At each node, it considers only:
\begin{itemize}
    \item The \textbf{natural neighbor} (the node directly in the direction of travel).
    \item \textbf{Forced neighbors} that arise when an obstacle forces a deviation.
\end{itemize}
This selective exploration is depicted in Figure~\ref{fig:pruning}. Notice how the natural neighbor is maintained, while forced neighbors are added only when obstacles are present.

\begin{figure}[H]
    \centering
    \includegraphics[width=0.6\textwidth]{figure2.png}
    \caption{Neighbor pruning in action. The algorithm retains the natural neighbor while adding forced neighbors when obstacles block the natural path.}
    \label{fig:pruning}
\end{figure}

\subsection*{Jump Points}
When an obstacle blocks the natural path, the algorithm creates a \textbf{jump point} at the obstacle's corner. This jump point resets the search direction and allows the algorithm to “jump” over multiple nodes instead of expanding each one. Figure~\ref{fig:jump} shows a typical jump point scenario, helping the search resume efficiently after encountering an obstacle.

\begin{figure}[H]
    \centering
    \includegraphics[width=0.8\textwidth]{figure3.png}
    \caption{When an obstacle is encountered, a jump point is introduced at its corner, allowing the algorithm to bypass unnecessary nodes and resume the search efficiently.}
    \label{fig:jump}
\end{figure}

\subsection*{How It All Comes Together}
The \texttt{jump()} function is central to the algorithm. It continues moving in the current direction until one of the following occurs:
\begin{itemize}
    \item The goal is reached.
    \item A forced neighbor is encountered.
    \item The direction of movement changes (e.g., from vertical to horizontal).
\end{itemize}
This mechanism minimizes unnecessary work and leads to optimal paths with far fewer node expansions. Figure~\ref{fig:benchmarks} provides benchmark results from the paper that demonstrate the efficiency improvements of JPS4 over A* in various map environments.

\begin{figure}[H]
    \centering
    \includegraphics[width=0.8\textwidth]{figure4.png}
    \caption{Benchmark results illustrating the speedup of JPS4 over A* across different map types, particularly in obstacle-dense environments.}
    \label{fig:benchmarks}
\end{figure}

\section{Comparison with Other Methods}
Compared to A*, JPS4 avoids unnecessary node expansions, resulting in a smaller open list and faster performance in cluttered maps. Unlike JPS8—which is designed for 8-connected grids—JPS4 refines these ideas specifically for grids limited to cardinal directions, making it more suitable for certain applications.

\section{Data Structures and Techniques}
JPS4 leverages several key data structures and techniques:
\begin{itemize}
    \item \textbf{Grid Representation:} A uniform-cost grid where each cell is either free or blocked.
    \item \textbf{Open List:} Managed using a priority queue (e.g., binary heap) to quickly select the lowest-cost node.
    \item \textbf{Pruning Mechanism:} Custom logic based on canonical ordering significantly reduces the search space.
\end{itemize}
These structures are optimized to work together efficiently and support the selective expansion approach of the algorithm.

\section{Implementation Outlook}
Although the theoretical design of JPS4 is robust, its practical implementation presents several challenges:
\begin{itemize}
    \item \textbf{Recursion Overhead:} The recursive nature of the \texttt{jump()} function might lead to deep call stacks. We could mitigate this by setting a recursion limit or using an iterative approach.
    \item \textbf{Edge Case Handling:} Special attention is needed for narrow corridors and complex obstacle configurations to ensure the algorithm finds optimal paths.
    \item \textbf{Memory Efficiency:} Efficient use of data structures for the grid and the open list is essential, especially when dealing with large maps.
\end{itemize}
Early strategies include optimizing priority queue operations and employing visualization tools to debug and verify the algorithm’s behavior.

\section*{Conclusion}
JPS4 offers a clever adaptation of jump point search for 4-connected grids by using canonical ordering and selective neighbor expansion. This results in significant performance gains in cluttered environments. While challenges such as recursion depth and edge-case handling remain, the potential improvements in efficiency make JPS4 a promising approach for applications in video games and robotics.

\end{document}